% Options for packages loaded elsewhere
\PassOptionsToPackage{unicode}{hyperref}
\PassOptionsToPackage{hyphens}{url}
%
\documentclass[
]{article}
\usepackage{amsmath,amssymb}
\usepackage{lmodern}
\usepackage{iftex}
\ifPDFTeX
  \usepackage[T1]{fontenc}
  \usepackage[utf8]{inputenc}
  \usepackage{textcomp} % provide euro and other symbols
\else % if luatex or xetex
  \usepackage{unicode-math}
  \defaultfontfeatures{Scale=MatchLowercase}
  \defaultfontfeatures[\rmfamily]{Ligatures=TeX,Scale=1}
\fi
% Use upquote if available, for straight quotes in verbatim environments
\IfFileExists{upquote.sty}{\usepackage{upquote}}{}
\IfFileExists{microtype.sty}{% use microtype if available
  \usepackage[]{microtype}
  \UseMicrotypeSet[protrusion]{basicmath} % disable protrusion for tt fonts
}{}
\makeatletter
\@ifundefined{KOMAClassName}{% if non-KOMA class
  \IfFileExists{parskip.sty}{%
    \usepackage{parskip}
  }{% else
    \setlength{\parindent}{0pt}
    \setlength{\parskip}{6pt plus 2pt minus 1pt}}
}{% if KOMA class
  \KOMAoptions{parskip=half}}
\makeatother
\usepackage{xcolor}
\usepackage[margin=1in]{geometry}
\usepackage{graphicx}
\makeatletter
\def\maxwidth{\ifdim\Gin@nat@width>\linewidth\linewidth\else\Gin@nat@width\fi}
\def\maxheight{\ifdim\Gin@nat@height>\textheight\textheight\else\Gin@nat@height\fi}
\makeatother
% Scale images if necessary, so that they will not overflow the page
% margins by default, and it is still possible to overwrite the defaults
% using explicit options in \includegraphics[width, height, ...]{}
\setkeys{Gin}{width=\maxwidth,height=\maxheight,keepaspectratio}
% Set default figure placement to htbp
\makeatletter
\def\fps@figure{htbp}
\makeatother
\setlength{\emergencystretch}{3em} % prevent overfull lines
\providecommand{\tightlist}{%
  \setlength{\itemsep}{0pt}\setlength{\parskip}{0pt}}
\setcounter{secnumdepth}{-\maxdimen} % remove section numbering
\ifLuaTeX
  \usepackage{selnolig}  % disable illegal ligatures
\fi
\IfFileExists{bookmark.sty}{\usepackage{bookmark}}{\usepackage{hyperref}}
\IfFileExists{xurl.sty}{\usepackage{xurl}}{} % add URL line breaks if available
\urlstyle{same} % disable monospaced font for URLs
\hypersetup{
  pdftitle={README},
  pdfauthor={Eric Zhao},
  hidelinks,
  pdfcreator={LaTeX via pandoc}}

\title{README}
\author{Eric Zhao}
\date{2023-01-23}

\begin{document}
\maketitle

\hypertarget{uncertainty-in-figures}{%
\subsection{Uncertainty in Figures}\label{uncertainty-in-figures}}

\hypertarget{histograms}{%
\subsubsection{Histograms}\label{histograms}}

In the first row, I produce a sequence of three histograms fixing the
number of estimate at 1000 and varying the number of rolls per estimate
from 10, 100,to 1000. We can notice that when number of rolls is fairly
small(N\_roll=10), distributions of P(d1+d2+d3=10) varies a lot and we
can't even see the red dashed lines representing mean within two sds. As
we increase the number of rolls to 100 and 1000, we notice that
distribution of probabilities starts to converge--the bounds become much
smaller and shape of distribution become more closer to normal
distribution.

In the second row, I produce a sequence of three histograms fixing the
number of rolls at 1000 and varying the number of estimate from 10, 100
to 1000. We observe that sd and mean is already settled in the first
place. The only difference among different number of estimates is the
shape of distribution since the sample size of estimates increases.

\hypertarget{stats}{%
\subsubsection{Stats}\label{stats}}

In fist set, mean gradually settles at 0.125 and standard deviations get
much smaller as N\_rolls increases. The quantiles also reflect the
changes but the results are not easy to tell.

For the second set, we can see both mean and standard deviations didn't
change much since the first graph.

\hypertarget{comparison-of-experiments}{%
\subsection{Comparison of Experiments}\label{comparison-of-experiments}}

Comparing the two sets of experiments, we can see the differences in
evolution of the distribution by varying different parameters. First,
the first row have both shape and bound change compared to second row in
which the bound of distribution didn't change much as number of
estimates increases.

The reason I think is that number of rolls determine the bound of the
probabilities whereas number of estimate determines the shape of the
probability estimate distributions.

\end{document}
